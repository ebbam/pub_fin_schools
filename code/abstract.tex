 Wage growth is a key driver of local wealth accumulation, enabling greater household and community investment in public goods. Unfortunately, wage growth has diverged markedly from productivity growth in the United States in recent decades. As such, regions whose wages track productivity gains tend to benefit from broader economic growth, while lagging regions risk weaker savings capacity and declining support for public services. This work estimates the effect of local wage shocks on education spending using a shift-share instrumental variable design, exploiting variation in commuting zone industrial composition interacted with national industry growth shocks. We estimate that a 10\% increase in local wages generates a 2.3\% short-run increase in per-pupil education expenditure, with a long-run effect of approximately 4.6\%. This result is only driven by a handful of states, whereas others exhibit little wage responsiveness. These results expedite the need to enable education equalisation programmes to account for future potential wage disparities.